% Options for packages loaded elsewhere
\PassOptionsToPackage{unicode}{hyperref}
\PassOptionsToPackage{hyphens}{url}
%
\documentclass[
]{book}
\usepackage{amsmath,amssymb}
\usepackage{lmodern}
\usepackage{iftex}
\ifPDFTeX
  \usepackage[T1]{fontenc}
  \usepackage[utf8]{inputenc}
  \usepackage{textcomp} % provide euro and other symbols
\else % if luatex or xetex
  \usepackage{unicode-math}
  \defaultfontfeatures{Scale=MatchLowercase}
  \defaultfontfeatures[\rmfamily]{Ligatures=TeX,Scale=1}
\fi
% Use upquote if available, for straight quotes in verbatim environments
\IfFileExists{upquote.sty}{\usepackage{upquote}}{}
\IfFileExists{microtype.sty}{% use microtype if available
  \usepackage[]{microtype}
  \UseMicrotypeSet[protrusion]{basicmath} % disable protrusion for tt fonts
}{}
\makeatletter
\@ifundefined{KOMAClassName}{% if non-KOMA class
  \IfFileExists{parskip.sty}{%
    \usepackage{parskip}
  }{% else
    \setlength{\parindent}{0pt}
    \setlength{\parskip}{6pt plus 2pt minus 1pt}}
}{% if KOMA class
  \KOMAoptions{parskip=half}}
\makeatother
\usepackage{xcolor}
\usepackage{longtable,booktabs,array}
\usepackage{calc} % for calculating minipage widths
% Correct order of tables after \paragraph or \subparagraph
\usepackage{etoolbox}
\makeatletter
\patchcmd\longtable{\par}{\if@noskipsec\mbox{}\fi\par}{}{}
\makeatother
% Allow footnotes in longtable head/foot
\IfFileExists{footnotehyper.sty}{\usepackage{footnotehyper}}{\usepackage{footnote}}
\makesavenoteenv{longtable}
\usepackage{graphicx}
\makeatletter
\def\maxwidth{\ifdim\Gin@nat@width>\linewidth\linewidth\else\Gin@nat@width\fi}
\def\maxheight{\ifdim\Gin@nat@height>\textheight\textheight\else\Gin@nat@height\fi}
\makeatother
% Scale images if necessary, so that they will not overflow the page
% margins by default, and it is still possible to overwrite the defaults
% using explicit options in \includegraphics[width, height, ...]{}
\setkeys{Gin}{width=\maxwidth,height=\maxheight,keepaspectratio}
% Set default figure placement to htbp
\makeatletter
\def\fps@figure{htbp}
\makeatother
\setlength{\emergencystretch}{3em} % prevent overfull lines
\providecommand{\tightlist}{%
  \setlength{\itemsep}{0pt}\setlength{\parskip}{0pt}}
\setcounter{secnumdepth}{5}
\usepackage{booktabs}
\usepackage{amsthm}
\makeatletter
\def\thm@space@setup{%
  \thm@preskip=8pt plus 2pt minus 4pt
  \thm@postskip=\thm@preskip
}
\makeatother
\ifLuaTeX
  \usepackage{selnolig}  % disable illegal ligatures
\fi
\usepackage[]{natbib}
\bibliographystyle{apalike}
\IfFileExists{bookmark.sty}{\usepackage{bookmark}}{\usepackage{hyperref}}
\IfFileExists{xurl.sty}{\usepackage{xurl}}{} % add URL line breaks if available
\urlstyle{same} % disable monospaced font for URLs
\hypersetup{
  pdftitle={Introduction to Statistics: an integrated textbook and workbook using R},
  pdfauthor={Sean Raleigh, Westminster College (Salt Lake City, UT)},
  hidelinks,
  pdfcreator={LaTeX via pandoc}}

\title{Introduction to Statistics: an integrated textbook and workbook using R}
\author{Sean Raleigh, Westminster College (Salt Lake City, UT)}
\date{2022-07-18}

\begin{document}
\maketitle

{
\setcounter{tocdepth}{1}
\tableofcontents
}
\hypertarget{intro}{%
\chapter*{Introduction}\label{intro}}
\addcontentsline{toc}{chapter}{Introduction}

Welcome to statistics!

If you want, you can also download this book as a PDF or EPUB file. Be aware that the print versions are missing some of the richer formatting of the online version. Besides, the recommended way to work through this material is to download the R notebook file (\texttt{.Rmd}) at the top of each chapter and work through it in RStudio.

\hypertarget{intro-philosophy}{%
\section*{Philosophy}\label{intro-philosophy}}
\addcontentsline{toc}{section}{Philosophy}

{[}TODO{]}

\hypertarget{intro-structure}{%
\section*{Course structure}\label{intro-structure}}
\addcontentsline{toc}{section}{Course structure}

{[}TODO{]}

\hypertarget{intro-onward}{%
\section*{Onward and upward}\label{intro-onward}}
\addcontentsline{toc}{section}{Onward and upward}

I hope you enjoy the textbook. You can provide feedback two ways:

\begin{enumerate}
\def\labelenumi{\arabic{enumi}.}
\item
  The preferred method is to file an issue on the Github page: \url{https://github.com/VectorPosse/Intro_Stats/issues}
\item
  Alternatively, send me an email: \href{mailto:sraleigh@westminstercollege.edu}{\nolinkurl{sraleigh@westminstercollege.edu}}
\end{enumerate}

\hypertarget{intro-r}{%
\chapter{Introduction to R}\label{intro-r}}

\hypertarget{intro-R-first-section}{%
\section{First section}\label{intro-R-first-section}}

  \bibliography{book.bib,packages.bib}

\end{document}
